\documentclass[oneside,a4paper,12pt]{article} % Gibt an: Papierformat, Schriftgröße

\usepackage{thesis}

% Hier werden die Abkürzungen definiert. Sofern ein Abkürzungsverzeichnis verwendet wird bitte entkommentieren.
%\newacronym{aifb}{AIFB}{Institut für Angewandte Informatik und Formale Beschreibungsverfahren}
\newacronym{kit}{KIT}{Karlsruher Institut für Technologie}
\newacronym[plural={Knowledge Graphs}]{kg}{KG}{knowledge graph}
\newacronym{bpe}{BPE}{byte pair encoding}
\newacronym{test}{TEST}{test}
\newacronym[plural={large language models}]{llm}{LLM}{large language model}
\newacronym{nlp}{NLP}{natural language processing}
\newacronym{kgqa}{KGQA}{knowledge graph question answering}
\newacronym[plural={natural language questions}]{nlq}{NLQ}{natural language question}
\newacronym[plural={pre-trained language models}]{plm}{PLM}{pre-trained language model}
\newacronym{lm}{LM}{language modeling}
\newacronym{nlu}{NLU}{natural language understanding}
\newacronym{nlg}{NLG}{natural language generation}


\begin{document}
\setlanguageGerman % Sprache einstellen.

% Hier kommt der ganze Vorspann. Bei Verwendung von Abkürzungs-, Abbildungs- oder Tabellenverzeichnisse bitte in dieser Datei entsprechend entkommentieren.
\include{sections/preamble}

% Füge hier die Kapitel per Verweis ein. Der Befehl \include fügt die angegebene tex-Datei an der jeweiligen Stelle ein. Die eingefügte Datei wird als normaler Teil des Quelltextes mitverarbeitet und ist daher LaTeX-Code (ohne Vorspann und \begin{document}...\end{document}).
\section{Überschrift1}
\label{Label1}

Hier steht Text. Diese Arbeit zitiert eine Beispielpublikation \cite{example_reference}.

% Beispiel für die Verwendung eines Abkürzungsverzeichnisses. Bitte entkommentieren:
%Das \gls{aifb} gehört zum \gls{kit}.

\subsection{Überschrift2}
\label{Label2}

Hier steht Text.

\section{Überschrift1}
\label{Label1}

Hier steht Text.

\subsection{Überschrift2}
\label{Label2}

Hier steht Text.

\include{sections/section3}
% ...usw.



% Anhang / Appendix
\appendix % Ab hier wird mit A, B, ... weiternummeriert.
% Für den Fall eines Anhangs entkommentieren
%\include{sections/appendix}

% Literaturverzeichnis
\bibliographystyle{acm}
\bibliography{YOUR_thesis} % Datei mit Literaturangaben einbinden

% Schriftliche Erklärung
\newpage
\include{sections/assertion}

\end{document}