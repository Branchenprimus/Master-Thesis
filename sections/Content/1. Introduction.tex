\section{Introduction}
\subsection{Background and Motivation}
Although the term \gls{kg} has appeared in the literature since at least 1972 \cite{schneider_course_1973}, its contemporary usage originates from Google's 2012 introduction of the Knowledge Graph \cite{singhal_introducing_2012}.

\glspl{kg} are one of the key success factors of many data-driven companies. Google, Facebook, eBay, and IBM, for example, develop \glspl{kg} to serve distinct yet overlapping purposes. Although all these knowledge graphs share a foundational reliance on structured data, their applications diverge based on industry needs, whether to improve search precision, enrich social interactions, improve product discovery, or enable enterprise-level knowledge extraction and reasoning. \cite{aggourCompoundKnowledgeGraphEnabled2022IntegrMaterManufInnov} 

%findig the right query to recieve the desired information out of a knowledge graph is however complex and requires knowledge about the enteties and structure of the graph itself

%Here come LLM into place which might be a solution for automated query generation

%The motiviation of this work lies in reducing the efford of getting the right information out of a knowlddge graph by leveraging llms

\subsection{Research Problem and Gap}

%Try to find paper, that lay down the current research on automated SPARQL query generation
%Find the shape literature, that is relevant for shape extraction
\subsection{Research Questions}


\subsection{Research Goals and Objectives}
\subsection{Structure of the Thesis}